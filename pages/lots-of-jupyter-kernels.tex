\chapter{Lots of Jupyter Kernels}
\section{Get jupyter kernel paths}

Jupyter notebook kernel definition are lived in data section. Use following 
command to all PATH definitions, and a possible location for kernel 
specification is \mintinline{bash}{$HOME/.local/share/jupyter/kernels}

\begin{minted}{bash}
# Example jupyter path result
➜  ~ jupyter --paths
config:
    /home/kaidong/.jupyter
    /home/kaidong/.local/etc/jupyter
    /usr/etc/jupyter
    /usr/local/etc/jupyter
    /etc/jupyter
data:
    /home/kaidong/.local/share/jupyter
    /usr/local/share/jupyter
    /usr/share/jupyter
runtime:
    /home/kaidong/.local/share/jupyter/runtime
\end{minted}

\section{Add jupyter ruby kernel}

Install \href{https://github.com/SciRuby/iruby}{IRuby} gem, and register 
IRubyKernel using following command:

\begin{minted}{bash}
  gem install iruby
  iruby register # add --force to overwrite
\end{minted}

\section{Add jupyter R kernel}

First install latest R using homebrew.

\begin{minted}{bash}
  brew install r
\end{minted}

And install \href{https://github.com/IRkernel/IRkernel}{IRkernel}, by running 
following command in R console:

\begin{minted}{R}
  install.packarges('IRkernel')
  IRkernel::installspec(name = 'ir43', displayname = 'R 4.3')
\end{minted}

By default, \mintinline{R}{IRkernel::installspec()} will install kernal with 
name "\rm(R)". Adding argument will result different kernel display inside 
jupyter lab kernel list.

\section{Add custom python environment kernel}

Making a duplication of origin python3 kernel with desired name with 
\mintinline{bash}{cp} (this will result the usage of default python3 logo).  
Then edit argv part. Here is an example of singularity python kernel:

\begin{minted}{json}
  {
   "argv": [
    "singularity",
    "exec",
    "--bind",
    "/home/kaidong:/home/kaidong",
    "--nv",
    "/path/to/source/torch.sif",
    "python3",
    "-m",
    "ipykernel_launcher",
    "-f",
    "{connection_file}"
   ],
   "display_name": "Torch (ipykernel)",
   "language": "python",
   "metadata": {
    "debugger": true
   }
 }
\end{minted}
